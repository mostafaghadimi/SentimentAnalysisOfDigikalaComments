
% -------------------------------------------------------
%  Abstract
% -------------------------------------------------------

\pagestyle{empty}
‌
\شروع{وسط‌چین}
\مهم{چکیده}
\پایان{وسط‌چین}
\بدون‌تورفتگی
در سال‌های اخیر رشد چشم‌گیر و قابل توجهی در مورد تحقیق درباره‌ی تحلیل احساسات از روی متن مشاهده شده است. تحلیل احساسات هم از جنبه‌ی کسب‌وکار و هم از نظر علمی حائز اهمیت است.
با روی کار آمدن مدل‌های زبان از‌پیش‌آموزش‌دیده و هم‌چنین با ایجاد مدل‌های زبانی قدرتمند، عصر جدیدی را در زمینه پردازش زبان طبیعی (NLP) آغاز شده است. در میان این مدل‌ها، مدل‌های مبتنی بر مبدل‌ها مانند BERT به دلیل عملکرد فوق‌العاده خود، به طور افزاینده‌
ای محبوب شده‌اند.
با این حال، این مدل‌ها معمولاً بر روی زبان انگلیسی متمرکز هستند و زبان‌های دیگر را به مدل‌های چند زبانه با منابع محدود واگذار می‌کنند. برای زبان فارسی نیز تحقیقات در زمینه‌ی مدل تک‌زبانه‌ی فارسی انجام شده است.
پردازش زبان طبیعی به ما در انجام وظایف مختلفی کمک می‌کند. در این میان تجزیه و تحلیل نظرات متنی مردم به‌خصوص در شبکه‌های اجتماعی، کسب‌وکارهای اینترنتی و... از اهمیت و ارزش بالایی برخوردار است. 
در این پایان‌نامه با محوریت تحلیل عواطف و احساسات نظرات شرکت دیجی‌کالا، ابتدا چالش‌های پیش‌روی در مدیریت حجم بسیار زیاد داده مطرح شده و سپس روش‌هایی برای تایید خودکار نظرات به کمک هوش مصنوعی ارائه شده است. برای استخراج احساسات با کمک مدل 
BERT
 از یک روش خلاقانه با تعداد شاخص‌های مشخص استفاده شده است که عمل‌کرد و خروجی شبکه را تا حد بسیار زیادی بهبود می‌دهد. تاثیرگذاری این مدل در شرکت دیجی‌کالا به نحوی بوده که هوش مصنوعی سهم ۹۰ درصدی از رد یا تایید نظرات را به خود اختصاص داده است.
	
\پرش‌بلند
\\
\بدون‌تورفتگی \مهم{کلیدواژه‌ها}: 
یادگیری ماشین، پردازش زبان طبیعی، دسته‌بندی متن، تجزیه و تحلیل احساسات، کلان‌داده، دیجی‌کالا
\صفحه‌جدید
